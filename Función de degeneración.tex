\documentclass[a4paper,10pt]{article}
\usepackage[spanish,activeacute]{babel}
\usepackage[utf8]{inputenc}
\usepackage{amsfonts,amsmath,amsthm,amssymb}


\title{Función de multiplicidad para osciladores armónicos}
\author{Leonardo Sanhueza M.}

\begin{document}

\maketitle

El problema del modelo del sistema binario, es el problema más simple en donde se hay una solución exacta conocida para la función de multiplicidad. Existe otro problema bastante sencillo de resolver. Éste corresponde al oscilador armónico, para el cual la solución fue originalmente obtenida por Max Planck. A menudo la derivación original puede resultar bastante engorrosa, pero para un estudiante que recién comienza con los estudios de ésta no necesita preocuparse de esta derivación pues existe un camino alternativo moderno, el cual es obtenido en el capítulo 4 y se muestra de una manera bastante simple. \\

Para comenzar con la obtención de la función de multiplicidad consideremos los estados cuánticos de un oscilador armónico, que tiene autovalores de energía:

\begin{equation}\label{eq1}
\epsilon_{s} = s \hbar \omega 
\end{equation} 

Donde $s$, $\omega$  corresponden al número cuántico que puede ser un entero positivo o cero y a la frecuencia angular del oscilador, respectivamente. Tener en cuenta que el número de estados es infinito, y la multiplicidad de cada uno es $1$. \\

Ahora, consideremos un sistema que contenga $N$ osciladores con la misma frencuencia. Queremos encontrar el número de caminos con el cual se obtiene la energía total de excitación, esto es:

\begin{equation}\label{eq2}
\epsilon = \sum_{i=1}^{N} s_{i} \hbar \omega = n \hbar \omega 
\end{equation}

Tener en cuenta que ésta energía puede ser distribuida entre los osciladores. Es decir, queremos encontrar la función de multiplicidad $g(N,n)$ de los N osciladores. Una observación importante es que a función de multiplicidad de los osciladores no es la misma que la función de multiplicidad del spin encontrada anteriormente. \\

Consideremos la función de "un" oscilador, para el cual se cumple que $g(1,n)=1$ para todos los valores de los números cuánticos $s$, el que es indéntico a n. Para resolver la ecuación \ref{eq5} de más abajo, necesitamos una función que represente o genere la serie:

\begin{equation}\label{eq3}
\sum_{n=0}^{\chi} g(1,n) t^n = \sum_{n=0}^{\chi} t^n 
\end{equation}

Es posible observar que se hace uso de una variable $t$, el cuál es sólo una herramienta temporal que será de gran ayuda para encontrar el resultado de \ref{eq5}, pero éste no aparece en el resultado final. 	Entonces la respuesta a lo propuesto es:

\begin{equation}\label{eq4}
\frac{1}{1-t} =  \sum_{n=0}^{\chi}  t^n
\end{equation}

Siempre y cuando se asuma que $|t | < 1$. Por lo tanto para el problema de $N$ osciladores, la función de generación es:

\begin{equation}\label{eq5}
\left( \frac{1}{1-t} \right)^N =  \left( \sum_{s=0}^{\chi}  t^s \right)^N = \sum_{n=0}^{\chi} g(N,n) t^n 
\end{equation}

pues, la cantidad de formas en que aparece el término $t^n$ en el producto de $N-\text{veces}$ es precisamente la cantidad de formas ordenadas para el cual el entero n se puede formar como la suma de los enteros N no negativos. \\

Observemos que,

\begin{eqnarray}\label{eq6}
g(N,n) &=& \lim_{n \to 0} \frac{1}{n!} \left( \frac{d}{dt} \right)^n \sum_{s=0}^{\chi} g(N,s)t^s \\
&=& \lim_{n \to 0} \frac{1}{n!} \left( \frac{d}{dt} \right)^n (1-t)^-N \\
&=& \frac{1}{n!} N (N+1) (N+2)... (N+n-1)
\end{eqnarray}

Por lo tanto, para el sistema de osciladores,

\begin{equation}
g(N,n) = \frac{(N+n-1)!}{n!(N-1)!}
\end{equation}

Este resultado se necesitará para resolver un problema en el siguiente capítulo.
\medskip

\begin{thebibliography}{X}
\bibitem{1} \textsc{Kittel} y \textsc{Kroemer},
\textit{Thermal Physics}, second edition,
New York, United State of America, 1980.

\end{thebibliography}
\end{document}
